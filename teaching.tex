\documentclass{article}
\usepackage{aas_macros}
\usepackage[letterpaper, margin=1in]{geometry}

\usepackage{etoolbox}
\newtoggle{UW}
%\toggletrue{UW}
\iftoggle{UW}{
    \newcommand{\where}{University of Washington }
}{
\newcommand{\where}{University of Michigan }
}

\makeatletter
\renewenvironment{thebibliography}[1]
     {%\section*{\refname}% <--- outcommented
      \@mkboth{\MakeUppercase\refname}{\MakeUppercase\refname}%
      \list{\@biblabel{\@arabic\c@enumiv}}%
           {\settowidth\labelwidth{\@biblabel{#1}}%
            \leftmargin\labelwidth
            \advance\leftmargin\labelsep
            \@openbib@code
            \usecounter{enumiv}%
            \let\p@enumiv\@empty
            \renewcommand\theenumiv{\@arabic\c@enumiv}}%
      \sloppy
      \clubpenalty4000
      \@clubpenalty \clubpenalty
      \widowpenalty4000%
      \sfcode`\.\@m}
     {\def\@noitemerr
       {\@latex@warning{Empty `thebibliography' environment}}%
      \endlist}
\makeatother

\date{}

\begin{document}
\title{Teaching Philosophy}
\author{Alex Kim}
\maketitle

%\section{Teaching}
In my current position at a National Laboratory I have not had teaching responsibilities.  However,
I have had the opportunity to guest lecture at UC Berkeley, lecture at summer schools, tutor friends
in continuing education, and direct undergraduate and graduate student research.  Cosmology
is a field that generates both lay and professional interest, giving me informal and formal opportunities to
educate.  I look forward to the rewards and challenges of a new role teaching as a professor. 

%goal: think like an expert
My principle goal in teaching is for students to appreciate the beauty that drew me to science: the quest to
describe  observed complexity with
theoretical simplicity.  Practically I want students to think as a professional scientist, being able to 
rationalize complex input information using simple explanations.
This thinking applies not only to the subject being taught,  but also
outside the classroom.

In my earlier teaching of physics,  I emphasized the commutative version
of the above-stated goal: that theoretical simplicity explains observed complexity.  I  emphasized the foundations of physics
rather than problem solving. As an experienced astrophysicist, I had come to
appreciate implicitly that theory is only successful if it solves problems.  Only by changing perspectives
did I appreciate that my students lacked the experience
to value theory on its own.  A  professor once told me
``You don't really understand the theory if you can't solve the problems,'' a philosophy that I now  appreciate and
apply in my own teaching.


% target student thinking and learning needs.
The teaching process  begins with understanding the students, identifying their goals and motivations, their abilities in critical thinking and
problem solving, and their preconceptions.  When a course first meets, I poll students on why they are taking the
class.  It is from their initial condition that I can chart the learning path toward the specific goals targeted
for the class. 

% motivate
Necessary are  motivated students who are willing to put in the commitment of time and effort
necessary to master solving the difficult problems in physics and astronomy.
In terms of the course,
it is important in the beginning to establish  to the students realistic goals and the path that the goals will be realized.
Students can be given partial ownership of the course by allotting time in which they can lead discussion and question the
professor.
In terms of the material, problems need to be motivated by real world problems.
This is particularly true when presenting advances that were made so far in the past, that it is not obvious
why the work was so exciting in its day. 
I like to give students a glimpse into what I as a scientist do day to day to show that coursework
is just a component of a path toward a rewarding career, in which day-to-day life bears little resemblance to what
happens in a classroom.

%implementation
My teaching goal is for the students to approach problems as an expert would, therefore
the valuable time I share with students is best spent having each side studying how the other
solves problems.  Both sides need to focus on breaking down the steps taken in problem solving: 
clarification of what question is  being asked,
isolation of relevant information, 
identification of the theory and models that apply, finding the simplest path toward solution,
going through the mechanics of solution, and confirmation that the result is reasonable.
My detailed play-by-play of my navigation though problem sets  allows students to observe how a scientist thinks on his
feet and my reliance on first, rather than derived, principles.  Conversely, observation of how students
approach challenging but doable problems will allow me to identify and confront common pitfalls they may make.
An interactive environment is needed as students ask me to make explicit cognitive steps I take for granted,
and for me to provide realtime feedback on their work and progress.
It is through intensive practice that expertise is earned, and it is in the classroom that students are effectively coached.
Homework is the primary vehicle through which students gain the information needed to solve problems
through reading assignments, and to gain more practice on their own time with assigned problem sets.

Independent of teaching philosophy, I will draw upon the wealth of experience, research, and teaching aids that are
available.
Colleagues and the internet have served as excellent resources for advice and ideas in preparing
my own lectures, while friends in the field of physics education keep me advised of current trends. 
 \iftoggle{UW}{The University of Washington Institute for Science & Mathematics Education}{I will avail
 myself to the resources for faculty at The Center
 for Research on Learning and Teaching.}

On a broader scope, I want to participate in increasing underrepresented minorities science majors.
Serving on the Lawrence Berkeley National Laboratory Diversity \& Inclusion Committee, one (of many) things I have
learned is the value of diversity in science.  Science is a social endeavor, increasingly so with
the increasingly larger collaborations formed to address big science.  The informational diversity
coming  from the inclusion of different  backgrounds and strengths fosters innovation.
Unclogging choke points at all levels of education is important.  I have tutored math at a predominantly African American
elementary school and at  \where \iftoggle{UW}{I plan to work on increasing undergraduate enrollment through the
Office of Minority Affairs \& Diversity Recruitment and Outreach.}{by being a 
mentor in the Michigan Louis Stokes Alliance for Minority Participation.}


\end{document}
