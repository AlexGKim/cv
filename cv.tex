\documentclass[line, margin]{res}

\usepackage{aas_macros}

\makeatletter
\renewenvironment{thebibliography}[1]
     {%\section*{\refname}% <--- outcommented
      \@mkboth{\MakeUppercase\refname}{\MakeUppercase\refname}%
      \list{\@biblabel{\@arabic\c@enumiv}}%
           {\settowidth\labelwidth{\@biblabel{#1}}%
            \leftmargin\labelwidth
            \advance\leftmargin\labelsep
            \@openbib@code
            \usecounter{enumiv}%
            \let\p@enumiv\@empty
            \renewcommand\theenumiv{\@arabic\c@enumiv}}%
      \sloppy
      \clubpenalty4000
      \@clubpenalty \clubpenalty
      \widowpenalty4000%
      \sfcode`\.\@m}
     {\def\@noitemerr
       {\@latex@warning{Empty `thebibliography' environment}}%
      \endlist}
\makeatother

\begin{document}
\name{Alex G. Kim}
\address{Lawrence Berkeley National Laboratory\\
1 Cyclotron Rd, MS: 50R6048\\
Berkeley, CA 94720-8164\\
agkim@lbl.gov
}
\begin{resume}

\section{Education}
Ph.D., Physics, University of California, Berkeley, Dec.\ 17 1996. {\it The Discovery of High-Redshift Supernovae and Their Cosmological Implications},
Richard Muller \& Saul Perlmutter advisors\\
M.A., Physics, University of California, Berkeley, 1994.\\
B.S., Physics, Mathematics, University of Michigan, Ann Arbor,1991.

\section{Professional Experience}
\begin{tabular}{lp{4.4in}}
2003-- & Staff Scientist, Physics Division,
Lawrence Berkeley National Laboratory\\
2002--2003 & Term Scientist, Physics Division,
Lawrence Berkeley National Laboratory\\
1999--2002 & Research Assistant, Center for Particle Astrophysics and Lawrence Berkeley National Laboratory\\
1997--1999& Research Associate, Laboratoire de Physique Corpusculaire et Cosmologie,
Coll{\`e}ge de France\\
1992--1997 & Research Assistant, Center for Particle Astrophysics and Lawrence Berkeley Laboratory\\
1989--1991 & Research Assistant, Physics Department, University of Michigan\\
\end{tabular}

\section{Honors and Awards}
\begin{tabular}{ll}
2007 & Cosmology Prize, Gruber Foundation\\
1991 & Phi Beta Kappa\\
1989--1990 & Northern Telecom Scholarship, Northern Telecom Inc., Vienna, Va.\\
1988--1990 & Franklin Tillery Scholarship\\
1988--1989 &  David Aspland Scholarship\\
1988 & William Branstrom Freshman Prize\\
1987--1991 & Angell Scholar\\
\end{tabular}

\section{Teaching Experience}
\begin{tabular}{lp{4.4in}}
2013 & University of California, Berkeley. Astro 260 Guest Lecturer\\
2000--2002 & Tutor, Malcolm X Elementary School, Berkeley CA\\
1991--1992 & University of California, Berkeley. Teaching assistant\\
\end{tabular}

\section{Professional Memberships}
American Astronomical Society

\section{Relevant Committees, and Service}
DOE National Lab Day Committee, 2014\\
LSST Cadence Workshop, Supernova Working Group co-Chair, 2014\\
LBNL Diversity and Inclusion Committee, 2014--\\
Distances Task Leader, Snowmass, 2013\\
DES Spring 2013 Collaboration Meeting, Local Organizing Chair, 2012--2013\\
Publications Committee, Nearby Supernova Factory, 2012--\\
SN Ia Working Group Leader, LSST DESC, 2012--\\
DES Spectroscopy Task Force Co-Chair, 2011--\\
DES Collaboration Meeting Steering Committee, 2011--\\
Referee for DOE-Office of Science proposals, 2010-- \\
Referee for Astrophysical Journal, Astroparticle Physics, Journal of Cosmology and Astroparticle Physics, Publications of the Astronomical Society of the Pacific, Publications of the Astronomical Society of Australia 1997--\\
Research supervisor for numerous students and postdocs, 1997--\\
Member of several doctoral thesis committees 2008--\\
SN Ia Spectroscopy Group leader, DES collaboration, 2008--2012\\
Joint Dark Energy Mission Interim Science Working Group, 2010--2011\\
Science Fair Judge, North School Hillsdale, CA Mar 2 2010\\
SN Ia Working Group leader, SNAP Collaboration, 2001--2009\\
Simulation Working Group leader, SNAP Collaboration, 1999--2009\\
System Managers Group member, SNAP Collaboration, 2001--2008\\
Review Presentation, Particle Physics Project Prioritization Panel (P5), Stanford, CA Friday, Feb. 22, 2008\\
Organizing Committe, Key Approaches to Dark Energy, Barcelona,  Aug, 2006\\
Internal Review Committee, Supernova Factory, 2004

\section{Research Activities}
1999--present --- Lawrence Berkeley National Laboratory
\begin{itemize}
\item Large Synoptic Survey Telescope - Dark Energy Science Collaboration --
\item Euclid Consortium -- Developing hardware and mission requirements for the SN~Ia/transient survey.
\item Dark Energy Survey -- Working on the optimization of the DES supernova survey and setting requirements
on external spectroscopic observations. Leading the spectroscopy component of the full collaboration.
\item Nearby Supernova Factory -- Improving the calibration of Type Ia supernovae as distance indicators.  Developing
a toolkit for generalized spectroscopy pipelines including high-performance computing and forward-modeling.
\item Joint Dark Energy Mission -- Developing science requirements, survey strategies, and mission configurations for
a NASA and DOE interagency satellite experiment intended to probe dark energy.  Participation as a member of the JDEM-Interim
Science Working Group charged with designing mission concepts for the two agencies.
\item Supernova Acceleration Probe (SNAP) Collaboration-- Developed and implemented a simulation package,
science requirements, survey strategies, and mission configurations for a proposed satellite experiment designed to probe dark energy.
Led two collaboration working groups.  Contributions led to the successful passing of scientific reviews and advocacy by
national committees.  Work on the supernova error budget and systematic uncertainties has resulted in several published papers with
more in progress.
\item Baryon Oscillation Spectroscopic Survey -- Developed and implemented the spectroscopic reduction pipeline for the BOSS experiment.  The software is currently processing nightly observations.
\item Supernova Cosmology Project -- Working on the statistical analysis for high-redshift supernova cosmology presented in
several papers.
\end{itemize}
1997--1999 --- Coll\`ege de France
\begin{itemize}
\item AGAPE -- Applied
 HST archival data to analyze the gravitational microlensing of unresolved
source stars within the AGAPE collaboration and to search for new events,  resulting in a published paper.
\item EROS -- Discovering and studying supernovae at low ($z<0.2$) and high ($z > 0.4$)
redshifts as part of the EROS supernova search and the Supernova Cosmology
Project.  The supernovae are used to measure cosmological parameters. Work has led to papers on
supernova rates and the cosmological distance scale.
\item Planck -- Developing algorithms and software simulating the data expected from
the Planck mission, a satellite that measures CMB temperature
anisotropies.  My focus was in determining hardware constraints, optimized scanning strategies, and data processing
methods for polarization measurements. Findings have been published.
\end{itemize}
1992--1997 --- Lawrence Berkeley National Laboratory
\begin{itemize}
\item Deepsearch (SCP) -- Discovered and studied over 28 distant supernovae, $0.35< z < 0.85$, in a
 project aimed at measuring the mass density of the universe and the
 Cosmological constant.  Involved in developing search software, observing,
 and data analysis.  Resulted in the discovery of the accelerated expansion of the Universe.
\end{itemize}

\section{Invited Talks}
Speaker, LSST Project and Community Workshop, Phoenix, AZ, Aug 14, 2014\\
Lecture, Santa Fe Cosmology Workshop, Santa Fe, NM, Jul 22, 2014\\
Speaker, DES-LSST Workshop, Fermi National National Laboratory, IL, Mar 26, 2014\\
Panel Discussion, Snowmass, Minneapolis MN, July 29, 2013\\
Seminar, LineA Seminar, Brazil, June 20, 2013\\
Colloquium, Herzberg Institute of Astronomy, Victoria, Canada, March 12, 2013\\
Lecture, IX Mexican School on Gravitation and Mathematical Physics, Puerto Vallarta, Mexico, Dec 5, 2012\\
Speaker, Korean Physical Society, Oct 26, 2012\\
Colloquium, Yonsei University, Oct 25, 2012\\
Seminar, Asia Pacific Center for Theoretical Physics, Pohang Korea, Oct 22, 2012\\
Speaker, 13th Marcel Grossmann Meeting, Stockholm, Sweden, July 1, 2012\\ 
Speaker, Nobel Prize Panel, University of Stockholm, Sweden, Nov 12, 2011\\
Lecturer, Azores School on Observational Cosmology, Angra do Heroismo, Portugal, Aug 31- Sep 6, 2011\\
Seminar, Astronomy Department, Yonsei University, Seoul, May 6, 2011\\
Seminar, Institute for the Early Universe, Ewha University, Seoul, May 3, 2011\\
Seminar, Korea Institute for Advanced Study, Seoul, May 2, 20011\\
Workshop, The Return of de Sitter, NORITA, Stockholm, March 15, 2011\\
Seminar, Center for Particle Astrophysics, Fermilab, May 17, 2010\\
Seminar, National Astronomy Observatory of China, Beijing, Apr 8 2010\\
Seminar, Institute of High Energy Physics, Beijing, Apr 7 2010\\
Seminar, Institute for the Early Universe, Ewha University, Seoul, Apr 5, 2010\\
Workshop, First Berkeley-Paris Dark Energy Cosmology Workshop, Paris, Sep 19, 2009\\
Conference, Frontiers of Cosmology at Dome A Antarctica, Suzhou, China, Jul 21, 2009\\
Speaker, Particle Physics Project Prioritization Panel (P5), Stanford, CA Friday, Feb. 22, 2008\\
Workshop, International Workshop on the Interconnection Between
Particle Physics and Cosmology, Texas A\&M, May 14, 2007 \\
Cosmology Seminar, University of California, Davis, Dec 7 2006\\
Workshop, Key Approaches to Dark Energy, Barcelona,  Aug, 2006\\
Workshop, European Dark Energy Network in Paris, Paris, Dec 9, 2005 \\
Workshop, Probing the Dark Universe with Subaru and Gemini, Waikoloa, HI, Nov 8, 2005\\
Colloquium, University of Missouri, Columbia, Oct 3, 2005\\
Seminar, Laboratoire d'Astrophysique de Marseille, Apr 3, 2003\\
Astrophysics Seminar, Univerity of California Riverside, Mar 24, 2003\\
Seminar, University of Florida, Mar 17, 2003\\
Colloquium, Florida Atlantic University, Mar 14, 2003\\
Astrophysics Seminar, Los Alamos National Laboratory, Feb 27, 2002\\
Colloquium, Purdue University, Oct 25, 2001\\
San Mateo Astronomical Society, May 3, 2001\\
Journal Club, Department of Astronomy, UC Berkeley, Apr 27, 2001\\
Colloquium, Coll\`ege de France, Mar 18, 2001\\
Colloquium, Centre de Physique des Particules de Marseille, Mar 15, 2001\\
Workshop, Frontiers in Contemporary Physics-II at Vanderbilt University, Mar 9, 2001\\
Colloquium, Indiana University, March 2001\\
Colloquium, Observatoire de Meudon, Feb 2000\\
Colloquium, Universidad de La Serena,
Chile, Sep 21, 1998\\
Conference,  XXXIIInd
Rencontres de Moriond ``Fundamental Parameters in Cosmology'',
Les Arcs 1800, France, Jan 17-24, 1998\\
Conference, NATO Advanced Study Institute on Thermonuclear Supernovae, Aiguablava,
Spain, June 20-30, 1995\\

\section{Refereed Publications}
\nocite{2014MNRAS.440.1498S}
\nocite{2014ApJ...784...51K}
\nocite{2013A&A...560A..90F}
\nocite{2013A&A...560A..66R}
\nocite{2013PhRvD..87l3512H}
\nocite{2013Fourspring}
\nocite{2013ApJ...770..108C}
\nocite{2013ApJ...770..107C}
\nocite{2013ApJ...766...84K}
\nocite{2013PhRvD..87b3520S}
\nocite{2012ApJ...753..152B}
\nocite{2012PhRvD..85l3530S}
\nocite{2011APh....34..847F}
\nocite{2011JCAP...06..020K}
\nocite{2011PASP..123..470S}
\nocite{2011PASP..123..230K}
\nocite{2010ApJ...716..712A}
\nocite{2010APh....33..248K}
\nocite{2009ApJ...700.1415N}
\nocite{2008ApJ...686..749K}
\nocite{2008ApJ...673..981K}
\nocite{2007APh....28..448K}
\nocite{2007A&A...470..411G}
\nocite{2007APh....27..213A}
\nocite{2006MNRAS.370..933J}
\nocite{2006ApJ...644....1C}
\nocite{2006PASP..118..205D}
\nocite{2006APh....24..451K}
\nocite{2005AJ....130.2788H}
\nocite{2005AJ....130.2278G}
\nocite{2005A&A...437..789N}
\nocite{2005A&A...430..843L}
\nocite{2004ApJ...615..595H}
\nocite{2004A&A...423..881B}
\nocite{2004AJ....128..387G}
\nocite{2004A&A...421..509A}
\nocite{2004MNRAS.347..909K}
\nocite{2004APh....20..377R}
\nocite{2003ApJ...598..102K}
\nocite{2003A&A...404..145A}
\nocite{2003MNRAS.340.1057S}
\nocite{2002ApJ...577..120P}
\nocite{2002PASP..114..803N}
\nocite{2002A&A...389L..69G}
\nocite{2002A&A...389..149D}
\nocite{2001A&A...378.1014A}
\nocite{2001ApJ...558..359G}
\nocite{2001A&A...373..126D}
\nocite{2000A&A...362..419H}
\nocite{2000ApJ...532..340A}
\nocite{2000A&AS..142..499R}
\nocite{2000A&A...355L..39L}
\nocite{2000PhST...85...47G}
\nocite{1999A&A...351L...5E}
\nocite{1999A&A...351...87E}
\nocite{1999A&A...348..175E}
\nocite{1999ApJ...517..565P}
\nocite{1999A&A...344L..63A}
\nocite{1999A&A...344L..49A}
\nocite{1998A&A...337L..17E}
\nocite{1998Natur.391...51P}
\nocite{1997ApJ...483..565P}
\nocite{1997ApJ...476L..63K}
\nocite{1996ApJ...473..356P}
\nocite{1996PASP..108..190K}
\nocite{1995ApJ...440L..41P}

\bibliography{pub}
\bibliographystyle{unsrt}

\end{resume}
\end{document}
