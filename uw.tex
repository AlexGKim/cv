\documentclass{article}
\usepackage{aas_macros}

\makeatletter
\renewenvironment{thebibliography}[1]
     {%\section*{\refname}% <--- outcommented
      \@mkboth{\MakeUppercase\refname}{\MakeUppercase\refname}%
      \list{\@biblabel{\@arabic\c@enumiv}}%
           {\settowidth\labelwidth{\@biblabel{#1}}%
            \leftmargin\labelwidth
            \advance\leftmargin\labelsep
            \@openbib@code
            \usecounter{enumiv}%
            \let\p@enumiv\@empty
            \renewcommand\theenumiv{\@arabic\c@enumiv}}%
      \sloppy
      \clubpenalty4000
      \@clubpenalty \clubpenalty
      \widowpenalty4000%
      \sfcode`\.\@m}
     {\def\@noitemerr
       {\@latex@warning{Empty `thebibliography' environment}}%
      \endlist}
\makeatother


\begin{document}
\title{Professional Interests}
\author{Alex Kim}
%\date{\color{green}December 2005}
\maketitle

I am an observational astrophysicist, with interests in
cosmology, supernovae, transients, and machine learning.
As a new faculty member, I will come to the University of Washington
primed as a key player in major collaborations, with a
track record in a diverse range of research.  I  value excellence
in teaching and research as part of the student educational experience
as a means of enriching our scientific community.

\section{Research Activity}
My principal field of research is  Type~Ia Supernova (SN~Ia) Cosmology. 
SNe~Ia are exploding stars which have known luminosity at peak brightness, and are  bright enough to be
seen to distances beyond 10 billion light years.  It is through the work
performed as part of my PhD thesis that accelerating
expansion of the Universe was discovered \cite{1997ApJ...483..565P}.  Today, SNe~Ia remain a key tool used
to measure the cosmic expansion history, which is used both to establish
the  backdrop within which the contents of the Universe evolve,
and to determine the nature of the new physics responsible
for the cosmic acceleration.
SN~Ia surveys are a pillar  in the community plans for the field of cosmology,
as laid out in
the Decadal Survey from the astrophysics and astronomy communities, and the Snowmass
\cite{2013arXiv1309.5386D,2013arXiv1309.5382K} and P5 reports
from the high-energy physics community.  I here detail my research activities,
which are critical to the success of high-redshift supernova cosmology.

\subsection{Type Ia Supernovae as Cosmological Distance Indicators}
Currently, the peak luminosities of SNe~Ia are not known, at least to the accuracies necessary
to achieve the lofty objectives of next generation experiments such as the Dark Energy Survey (DES) and the Large Synoptic
Survey Telescope (LSST).  The unique data set of
the Nearby Supernova Factory (SNfactory) collaboration consists of the spectrophotometric
time series, covering pre-maximum to 40-days post-maximum phases,
of  $\sim 150$ nearby ($0.03<z<0.08$) SNe~Ia.
I have used these data  to identify new
empirical
parameters based on synthetic broad-band photometry that improve reduce distance uncertainties from 7\% to 4.5\% \cite{2013ApJ...766...84K}
while eliminating systematic biases correlated with host-galaxy
properties \cite{2014ApJ...784...51K}.  This work represents only the beginning of the
potential information that can be mined from  SNfactory data: The next order of business
is to include spectroscopic information (rather than limiting ourselves to broadband photometry)
and apply non-linear dimensionality reduction to efficiently find and parameterize
subpopulations   within the SN~Ia family.  Tools usable by the community must be furnished
so that what we learn from the
exquisite data from the SNfactory can be applied to the lower-resolution, lower signal-to-noise data obtained
by DES and LSST.  I am currently working with a graduate student on producing the next-generation empirical
model of SN~Ia luminosities.

\subsection{Measurement of the Expansion History of the Universe With Imaging Surveys  DES and LSST}
The Dark Energy Survey and the Large Synoptic Survey Telescope represent the present and future of
SN~Ia cosmology: they
will generate the dense ground-based Hubble diagram with expanded redshift coverage that will serve as the standard 
through the 2020s.  However,
the most important and challenging product these surveys must provide is not the Hubble curve itself,
but rather its uncertainties: trusted error bars must accompany any experimental evidence for the exclusion
of the Cosmological
Constant. My research focuses on
minimizing and quantifying the
systematic uncertainties that will dominate the error budget from DES and LSST \cite{2004MNRAS.347..909K}.  

DES is currently in its second year of a five-year survey running at the 4-m Blanco Telescope in Chile.  It is performing
cadenced observations of 30 square degrees in order to discover and generate SN~Ia light curves out
to $z\sim1$.
The principal challenge for DES (and LSST) is that these surveys do
not provide all the information necessary for precision cosmology.
Unlike  supernovae found in the current literature, the vast majority of objects from DES
will not have spectroscopic classification as SN~Ia, nor will their
measurements have the signal-to-noise, restframe wavelength coverage, and wavelength resolution that allow 
the characterization of SN~Ia subclasses.  This classification is necessary to track redshift-dependent evolution of 
the SN~Ia population.  In DES, I  formulated the strategy for performing the cosmology analysis
with a rigorous quantification of systematic uncertainties.  These uncertainties are  derived from a subset of 
DES objects that receive intensive observations beyond what is standard;
bias is measured as the difference between distances inferred with
and without these extra data. 
DES plans to begin its cosmological analyses in ernest at the end
of this semester: I am now  working with the DES SN working group to implement my analysis strategy.

LSST  represents the future of ground-based wide-field imaging surveys.
A telescope with an effective 6.7-m will operate from 2022-2032, observing the available sky (18,000 square degrees over the course
of a year)
every few days.  In addition, a select $\sim 100$ square degrees will be observed with a higher cadence of every 4 days
in each of six bands.
A primary science product of LSST is the discovery and measurement of SN~Ia light curves.
The LSST-Dark Energy Science Collaboration (LSST-DESC) is preparing
to exploit  the public data products that will be
delivered by the  LSST Project;  I have identified and organized
work as co-chair of the SN~Ia science working group \cite{2012arXiv1211.0310L}.
The first order of business is to ensure that LSST itself delivers data that are as close to optimal for the science, though
the selections of filter vendor, survey fields, exposure times, observing cadence, and sky tiling strategy.
Working with the LSST Project (including UW colleagues), I have developed and implemented figures of merit for the resultant supernova science
\cite{LSSTCadence}.  
We are incorporating realistic SN~Ia models  into LSST Project simulations to create high-fidelity representations
of expected SN light curves, for use in survey optimization and algorithm development.  Progress on this ongoing work will be
facilitated by my being co-located with key players in the LSST Project.

I look forward to taking advantage of my participation in both DES and LSST through two new projects.
First, despite the temporal ordering of the two experiments, the LSST Data Management has had more support and therefore
already has a software base that is competitive with that of DES.  I plan to examine whether the LSST code that extracts photometry simultaneously from mulitband images
of the same source, can be modified for use on the photometric extraction of supernova signal from DES data.
Second, a LSST Project deliverable that is critical for supernova science is a pure and efficient stream of transients.
I plan to use
DES data, as the closest available representation of the Universe that LSST will observe, to develop the 
analysis of the transient stream.


\subsection{Spectroscopy to Complement Imaging Surveys}
Imaging surveys such as DES and LSST identify interesting targets through broadband photometry.
Detailed examination of these objects requires spectroscopy. Spectroscopic follow-up is not
built into the DES and LSST Projects themselves, and so must be obtained through independent means.
In DES we apply for competed time at
telescopes, and form external collaborations with colleagues with institutional access.  I am a founding member of the OzDES collaboration, specifically formed to obtain spectroscopy
of interesting DES targets using the AAOmega-2dF, a 400-fiber multi-object spectrograph which covers
3 square degrees, that is mounted on the Australian Astronomical Telescope.
The sources we observe cater ti a broad range of scientific interests,
including the classification of live transients, characterization of their host galaxies, confirmation
of strong gravitational lens systems, getting redshifts of galaxy clusters, monitoring of active galactic nuclei, photometric-redshift
calibration,
and
standard stars candidates.  My primary interest in OzDES is to obtain a flux-limited transient sample with which to understand the
underlying population of SNe~Ia  \cite{2006MNRAS.370..933J}, and interlopers that could contaminate DES sample.  OzDES has made first identification of DES-discovered supernova \cite{2013ATel.5568....1C}.
Additionally, I am interested in the discovery of new strong gravitational lens systems, and the potential use
of reverberation mapping of active galactic nuclei as a new cosmic distance indicator.


LSST will generate both quiescent and transient targets for spectroscopy.  I am convinced that
a major independent collaboration is called for, one that will have competitive advantage over those with the
publicly provided Project data.
I am therefore exploring possible resources that can contribute to LSST spectroscopy followup.
Just as OzDES is an almost perfect spectroscopic complement to  DES, 
DESI has the field of view and multiplexing scaled to LSST;
DESI is a 5000-fiber multi-object spectrograph covering $>3\deg$-diameter
field of view that will be mounted on the 4-m Mayall telescope at Kitt Peak National Observatory.   Within the DESI collaboration I have proposed observing LSST targets, both using unused fibers during the
DESI survey and in a dedicated bright-time survey
for unfettered use of all DESI fibers.

SN~Ia spectroscopic subclassification requires an integral field spectrograph (IFS) for
the robust photometry of supernovae that cannot be obtained with fiber or slit spectrographs:
indeed, the primary  instrument for supernova {\it photometry} in WFIRST is an IFS.  To take
advantage of LSST products, the community needs a ground-based
network of 2 to 10-m telescopes equipped with  IFS's at the ready to
point at LSST-discovered transients, which will span over a broad  brightness range.  I am currently exploring
programs for the design and building of spectrographs similar to that planned for WFIRST for installation on ground-based
observatories.  Such instruments can serve as the backbone for
new collaborations
for SN~Ia-specific spectroscopy of LSST discoveries. 

\subsection{Cosmology with Space-Based Surveys}
The next leap in supernova cosmology requires the stable platform, infrared coverage, and exquisite image
quality afforded by a space telescope.  I have played a leading role in the development
of science and system requirements, and the experimental and survey design of the proposed
and realized space missions SNAP, JDEM, and Euclid
\cite{2006PASP..118..205D, 2011PASP..123..470S, 2013Fourspring, 2014arXiv1409.8562A}.
A  high-redshift supernova program is one of the science 
drivers of
NASA's WFIRST-AFTA satellite.  A call for science team
proposals is anticipated  within the next two years.  As a member of the JDEM Interim Science Working Group, I designed the
supernova observing strategy now adopted by WFIRST, and I calculated the performance for SNAP's IFS which is similar to the one baselined for WFIRST.  Saul Perlmutter and I intend
to remount a prototype IFS built for SNAP as a testbed for
the WFIRST instrument.  Independently, I am contributing a SN~Ia Supernova section to a White Paper 
edited by David Spergel and Bhuvnesh
Jain, which details an expanded science case for WFIRST with specific emphasis
on synergies with LSST.  It is in these ways that I am positioning myself to be a key member of the winning WFIRST science
team.


\subsection{Novel Cosmological Probes}
Direction in our field is driven by a combination of community consensus and emergent ideas.
The research presented so far  lie  within the
mainstream plan laid out by the agencies funding wide-field imaging surveys
\cite{2013arXiv1309.5386D,2013arXiv1309.5382K}.
Nevertheless, scientific curiosity demands looking into speculative but potentially game-changing
techniques, just as we did back in the early days of high-redshift supernova cosmology.
I am intrigued with the potential of measuring cosmic distances using delays in the arrival times of light from multiple images
of strongly lensed quasars.  The a priori unknown intrinsic light curve of the source and differing microlensing
in the image paths complicate the determination of the time delay.  I developed a
non-linear regression method to simultaneously model the intrinsic light curve, independent foreground
stochastic processes, and fit the time delay \cite{2013PhRvD..87l3512H}.  With the discovery of more 
strongly-lensed quasars with DES and LSST, modeling of the lens-system mass distribution, and accurate
time delay measurements, strongly-lensed
quasars can emerge as the next cross-cutting probe of Dark Energy, highly complementary with existing methods.

Redshift drift (often referred to as the Sandage-Loeb test) provides the most direct  measurement
of the cosmic expansion as the measure of $\dot{z}$, the evolution in redshift of an object over a time interval.
The instrumental
and astrophysical challenge
in making such a measurement within a human lifetime
is that the small signal calls for $10^{-10}$ redshift precisions.  Intriguingly, the wavelength resolution needed to measure
cosmological redshift drift is similar to that needed for the discovery of planets through stellar doppler shifts.
Interferometric spectrographs designed for planet searches may provide the redshift accuracies for emission-line galaxies
that allow the measurement of redshift drift
\cite{drift}.
I  am examining whether the astrophysical and instrumental challenges in the feasibility of such a program can be overcome.
Ongoing  emission-line-galaxy  spectroscopic surveys such
as eBOSS, DESI, and the spatially-resolved CALIFA will identify and characterize
galaxies that are well-suited for the measurement, and the building of prototype spectrographs optimized will demonstrate that
instrumental precisions and accuracies are achievable.

\subsection{Statistical Methods in Astrophysics}
The scientific program I have described often involves regression, classification, dimensionality reduction,
and machine learning, topics at the intersection of science, statistics, and computing.  I have applied Gaussian Process modeling
on a diverse range of applications, principal component
analysis (PCA) and expectation maximization PCA \cite{2012PhRvD..85l3530S,2013ApJ...766...84K,
2013PhRvD..87l3512H}, in some cases programming and running on high performance computers.
I am now pursuing non-linear dimensionality reduction, which can more efficiently encapsulate
the physical processes that govern SN~Ia light curves and spectra.
Diffusion maps provide a non-linear coordinate transformation based on the connectivity of a set of points.
I am now starting to explore its use, working with a graduate student on
to efficiently capture diversity in SN~Ia spectral time series, and for classification of large red galaxies in DES
and for the selection of emission line galaxies that will be targeted by DESI.

My interest in the application of statistics on cosmological data
was born out of interaction with statisticians, computer scientists,
and scientists. I have found that real connections are formed not only through discussion,
but by identifying new projects and the joint writing of new proposals.
I look forward to engaging in
the eScience Institute, which provides an excellent environment to develop new ideas and applications
for machine learning and astrophysics on high-performance computers.


\subsection{Collaborations}
I have referred to a number of collaborations of which I am a member, whose benefits of participation
and data  I will bring to UW.  For clarity, I list these collaborations and their dates of operation.

\begin{description}
\item[SNfactory (present -- 2015)]  Provides a unique dataset of
spectrophotometric time series of $0.03<z<0.08$ SN~Ia light curves.  The wealth of information
within the data are ripe for immediate analysis. The discoveries and results from SNfactory directly tied to the supernova
analysis of DES and LSST.
\item[DES (present -- 2018)] Currently collecting what will be the state-of-the-art deep/wide-field/multi-band
imaging survey.   The ``Stage III'' dark energy experiment, DES will provide the leading supernova
cosmology result in the near future.
\item[OzDES (present -- 2018)] Currently collecting thousands of spectra of interesting targets identified by DES,
filling an essential role in the spectroscopy needs.
\item[DESI (2018-- 2025)]  Primarily a galaxy and quasar redshift survey, the instrumentation is well matched for
getting spectroscopy of targets identified by LSST. Passed the first DOE funding review.
\item[LSST-DESC (2022 -- 2032)]  The ``Stage IV'' dark energy experiment, that will produce large streams of
transient discoveries and light curves, including SNe~Ia over a broad redshift range.
\end{description}

I am working to build new collaborations that are critical for the dark energy program, 
to
spectroscopically follow-up LSST targets, and to serve as the WFIRST science team.

\section{Teaching}
In my current position at a National Laboratory I have not had teaching responsibilities.  However,
I have had the opportunity to guest lecture at UC Berkeley, lecture at summer schools, tutor friends
in continuing education, and direct undergraduate and graduate student research.  Cosmology
is a field that generates both lay and professional interest, giving me informal and formal opportunities to
educate.  I look forward to the rewards and challenges of a new role teaching as a professor. 

%goal: think like an expert
My principle goal in teaching is for students to appreciate the beauty that drew me science: the quest to
describe  observed complexity with
theoretical simplicity.  Practically I want students to think as a professional scientist, being able to 
rationalize complex input information using simple explanations.
This thinking applies not only to the subject being taught,  but also
outside the classroom.

In my earlier teaching of physics,  I emphasized the commutative version
of the above-stated goal: that theoretical simplicity explains observed complexity.  I  emphasized the foundations of physics
rather than problem solving. As an experienced astrophysicist, I had come to
appreciate implicitly that theory is only successful if it solve problems.  Only by changing perspectives
did I appreciate that my students lacked the experience
to value theory on its own.  A  professor once told me
``You don't really understand the theory if you can't solve the problems,'' a philosophy that I now  appreciate and
apply in my own teaching.


% target student thinking and learning needs.
The teaching process  begins with understanding the students, identifying their goals and motivations, their abilities in critical thinking and
problem solving, and their preconceptions.  When a course first meets, I poll students on why they are taking the
class.  It is from their initial condition that I can chart the leaning path toward the specific goals targeted
for the class. 

% motivate
Necessary are  motivated students who are willing to put in the commitment of time and effort
necessary to master solving the difficult problems in physics and astronomy.
In terms of the course,
it is important in the beginning to establish  to the students realistic goals and the path that the goals will be realized.
Students can be given partial ownership of the course by allotting time in which they can lead discussion and question the
professor.,
In terms of the material, problems need to be motivated by real world problems.
This is particularly true when presenting advances that were made so far in the past, that it is not obvious
why the work was so exciting in its day. 
I like to give students a glimpse into what I as a scientist do day to day to show that coursework
is just a component of a path toward a rewarding career, in which day-to-day life bears little resemblance to what
happens in a classroom.

%implementation
My teaching goal is for the students to approach problems as an expert would, therefore
the valuable time I share with students is best spent having each side studying how the other
solves problems.  Both sides need to focus on breaking down the steps taken in problem solving: 
clarification of what question is is being asked,
isolation of relevant information, 
identification of the theory and models that apply, finding the simplest path toward solution,
going through the mechanics of solution, and confirmation that the results is reasonable.
My detailed play-by-play of my navigation though problem sets  allows students to observe how a scientist thinks on his
feet and my reliance on first, rather than derived, principles.  Conversely, observation of how students
approach challenging but doable problems will allow me to identify and confront common pitfalls they may make.
An interactive environment is needed as students ask me to make explicit cognitive steps I take for granted,
and for me to provide realtime feedback on their work and progress.
It is through intensive practice that expertise is earned, and it is in the classroom that students are effectively coached.
Homework is the primary vehicle through which students gain the information needed to solve problems
through reading assignments, and to gain more practice on their own time with assigned problem sets.

Independent of teaching philosophy, I will draw upon the wealth experience, research, and teaching aids that are
available.
Colleagues and the internet have served as excellent resources for advice and ideas in preparing
my own lectures, while friends in the field of physics eduction keep me advised of current trends. 

On a broader scope, I want to participate in increasing underrepresented minorities science majors.
Serving on the Lawrence Berkeley National Laboratory Diversity \& Inclusion Committee, one (of many) things I have
learned is the value of diversity in science.  Science is a social endeavor, increasingly so with
the increasingly larger collaborations formed to address big science.  The informational diversity
coming  from the inclusion of different  backgrounds and strengths fosters innovation.
Unclogging choke points at all levels of education is important.  I have tutored math at a predominantly African American
elementary school, and at UW I plan to work on increasing undergraduate enrollment through the
Office of Minority Affairs \& Diversity Recruitment and Outreach




\section{References: Select Publications}
The following references represent a selection from my publication record associated with the research
activities that I will engage in at the University of Washington.

\bibliographystyle{hplain}
\bibliography{pub}
\end{document}
