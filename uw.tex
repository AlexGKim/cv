\documentclass{article}
\usepackage{aas_macros}

\makeatletter
\renewenvironment{thebibliography}[1]
     {%\section*{\refname}% <--- outcommented
      \@mkboth{\MakeUppercase\refname}{\MakeUppercase\refname}%
      \list{\@biblabel{\@arabic\c@enumiv}}%
           {\settowidth\labelwidth{\@biblabel{#1}}%
            \leftmargin\labelwidth
            \advance\leftmargin\labelsep
            \@openbib@code
            \usecounter{enumiv}%
            \let\p@enumiv\@empty
            \renewcommand\theenumiv{\@arabic\c@enumiv}}%
      \sloppy
      \clubpenalty4000
      \@clubpenalty \clubpenalty
      \widowpenalty4000%
      \sfcode`\.\@m}
     {\def\@noitemerr
       {\@latex@warning{Empty `thebibliography' environment}}%
      \endlist}
\makeatother


\begin{document}
\title{Professional Interests}
\author{Alex Kim}
%\date{\color{green}December 2005}
\maketitle

I am an observational astrophysicist, with interests in
cosmology, supernovae, transients, and machine learning.
As a new faculty member, I will come to the University of Washington
primed as a key player in major collaborations, with a
track record in a diverse range of research.  I  value excellence
in teaching and research as part of the student educational experience
as a means of enriching our scientific community.

\section{Research Activity}
My principal field of research is  Type~Ia Supernova (SN~Ia) Cosmology. 
SNe~Ia are exploding stars that have known luminosity at peak brightness, and are sufficiently bright to be
seen even out to distances beyond 10 billion light years.  It is through the work
performed as part of my PhD thesis that accelerating
expansion of the Universe was discovered \cite{1997ApJ...483..565P}.  Today, SNe~Ia remain a key tool used in
the measurement of the cosmic expansion history, necessary both to establish
the  backdrop within which the contents of the Universe evolve,
and to determine the nature of dark energy, the new physics responsible
for gravitational repulsion and cosmic acceleration.
SNe~Ia surveys are a foundation in the community plan for the field of cosmology,
as laid out in
the Decadal Survey representing the Astrophysics and Astronomy community, and the Snowmass
\cite{2013arXiv1309.5386D,2013arXiv1309.5382K} and P5 reports
representing the high-energy physics community.

\subsection{Type Ia Supernovae as Cosmological Distance Indicators}
In reality, the peak luminosities of SNe~Ia are not known, at least to the accuracies necessary
to achieve the lofty objectives of next generation experiments such as Dark Energy Survey (DES) and the Large Synoptic
Survey Telescope (LSST).  The unique data set of
the Nearby Supernova Factory (SNfactory) collaboration consists of the spectrophotometric
time series, covering pre-maximum to 40-days post-maximum phases,
of  $\sim 150$ nearby ($0.03<z<0.08$) SNe~Ia.
I have used these data  to identify new
empirical
parameters based on synthetic broad-band photometry that improve reduce distance uncertainties from 7\% to 4.5\% \cite{2013ApJ...766...84K}
while eliminating systematic biases correlated with contextual information, i.e.\ host-galaxy
properties \cite{2014ApJ...784...51K}.  This work represents only the beginning of the
potential information that can be mined from  SNfactory data: The next order of business
is to include high-resolution spectroscopy (rather than  broadband photometry)
and apply non-linear dimensionality reduction to efficiently find and parameterize
subpopulations   within the SN~Ia family.  Critically, we must provide the community the tools
to apply everything we learn from the
exquisite data from the SNfactory to the lower-resolution, lower signal-to-noise data obtained
in high-redshift supernova surveys.

\subsection{Measurement of the Expansion History of the Universe With Imaging Surveys  DES and LSST}
The Dark Energy Survey and the Large Synoptic Survey Telescope represent the present and future of
SN~Ia cosmology: they
will generate the dense ground-based Hubble diagram with expanded redshift coverage that will be the standard
through the 2020s.  However,
the most important and challenging product these surveys must produce is not the Hubble curve itself,
but rather its uncertainties: trusted error bars must accompany any experimental evidence for the exclusion
of the Cosmological
Constant universe. My research focuses on
minimizing and quantifying the
Systematic uncertainties will dominate the error budget from DES and LSST \cite{2004MNRAS.347..909K}.  

DES is currently in its second year of a five-year survey running at the 4-m Blanco Telescope in Chile.  It is performing
cadenced observations of 30 square degrees in order to discover and generate SN~Ia light curves out
to $z\sim1$.
The principal challenge for DES (and LSST) is that these surveys do
not provide all the information necessary for precision cosmology.
Unlike  supernovae found in the current literature, the vast majority of objects from DES
will not have spectroscopic classification as SN~Ia., nor will their
measurements have the signal-to-noise, restframe wavelength coverage, and wavelength resolution that allow 
the characterization of SN~Ia subclasses.  This classification is necessary to track redshift-dependent evolution of 
the SN~Ia population.  In DES, I have formulated the strategy for performing the cosmology analysis
with a rigorous quantification of systematic uncertainties.  These uncertainties are precisely derived from a subset of 
DES objects that receive intensive observations beyond what is standard;
bias is measured as the difference between distances inferred with
and without those extra data. 
DES plans to begin its cosmological analyses in ernest at the end
of this semester: now is the time to implement my analysis strategy.

LSST  represents the future of ground-based wide-field imaging surveys.
A telescope with an effective 6.7-m will operate from 2022-2032, observing the available sky (18,000 square degrees over the course
of a year)
every few days.  In addition, a select 100 square degrees will be observed with a higher per-band cadence of every 4 days.
A primary science product of LSST is the discovery and measurement of SN~Ia light curves.
The LSST-Dark Energy Science Collaboration (LSST-DESC) has formed and is preparing
to exploit  the public data products that will be
delivered by the  LSST Project;  I have identified and organized
our work as co-chair of the SN~Ia science working group \cite{2012arXiv1211.0310L}.
The first order of business is to ensure that LSST itself delivers data that are as close to optimal for the science, though
choices of filter vendor, field selection, exposure times, observing cadence, and sky tiling strategies.
Working with the LSST Project (including UW colleagues), I have developed and implemented figures of merits for the resultant supernova science
\cite{LSSTCadence}.  
We are implementing realistic SN~Ia models for injection into LSST Project simulations to create high-fidelity representations
of LSST SN data products, to use for survey optimization and algorithm development.  This ongoing work will be
facilitated by being co-located with key players in the LSST Project.

My participation in both DES and LSST offers synergistic opportunities.  There are two examples I am particularly interested
in.  First, despite the sequence of the two experiments, the LSST Data Management has had more support and therefore
already has a software base that is competitive with that of DES.  Indeed, there are LSST implementations of
code that does not yet exist for DES; the LSST MultiFit, which extracts photometry simultaneously from mulitband images
of the same source, is almost ideal for the photometric extraction of supernova signal from DES data.
Second, a LSST Project deliverable that is critical for supernova science is a pure and efficient stream of transients.
DES data stand as the closest available representation of the Universe that LSST will observe and are therefore
essential in developing the LSST pipeline.

\subsection{Spectroscopy to Complement Imaging Surveys}
Imaging surveys such as DES and LSST identify interesting targets through broadband photometry.
Detailed examination of these objects requires spectroscopy. Spectroscopic follow-up is not
built into the DES and LSST Projects themselves, and so must be obtained through independent means.
In DES we apply for competed time at
telescopes and form external collaborations with colleagues with access to important resources.  I am a founding member of the OzDES collaboration, specifically formed to obtain spectroscopy
of interesting DES targets using the AAOmega-2dF 400-fiber multi-object spectrograph that covers
3 square degrees at the Australian Astronomical Telescope.
The sources we observe represent a broad range of scientific interests,
including the classification of live transient objects, characterization of transient host galaxies, confirmation
of strong gravitational lens systems, redshifting of galaxy clusters, monitoring of active galactic nuclei, photometric-redshift
calibration,
and
standard stars candidates.  My primary interest in OzDES is classifying a flux-limited transient sample to understand the
underlying population of SNe~Ia  \cite{2006MNRAS.370..933J}, and interlopers that could contaminate DES samples
without classification.  OzDES has made first identification of DES-discovered supernova \cite{2013ATel.5568....1C}.
Otherwise, I am interested in the discovery of new strong gravitational lens systems, and the potential use
of reverberation mapping of active galactic nuclei as a new cosmic distance indicator.


LSST will generate many targets for spectroscopy, both quiescent and transient objects.  I am convinced that
a major independent collaboration is called for, one that will have competitive advantage over those with the
public imaging data provided by the Project.
I am therefore exploring possible resources that can contribute to LSST spectroscopy followup.
Just as with OzDES and DES, the Dark Energy Spectroscopy Instrument
(DESI) experiment can follow LSST
targets.  DESI is a 5000-fiber multi-object spectrograph covering $>3\deg$-diameter
field of view that will be mounted on the 4-m Mayall telescope at Kitt Peak National Observatory.   Within the DESI collaboration, I have proposed the use of unused fibers during the
DESI survey, and the full resource of fibers during bright time, on LSST targets.

SN~Ia spectroscopic subclassification requires different instruments on a broader range of
telescope apertures than provided by DESI.  Integral field spectrographs are required for
the robust photometry of supernovae that cannot be obtained with fiber or slit spectrographs.  
Indeed, this is the main instrument for supernova in WFIRST.  
Collaboration building.

\subsection{Cosmology with Space-Based Surveys}
The next leap in supernova cosmology requires the stable platform, infrared coverage, and exquisite image
quality afforded by space-based observing.  I have played a leading role in the development
of science and system requirements, and the experimental and survey design of the proposed
and realized space missions SNAP, JDEM, and Euclid.
\cite{2006PASP..118..205D, 2011PASP..123..470S, 2013Fourspring, 2014arXiv1409.8562A}.
The WFIRST-AFTA satellite, which has a high-redshift supernova program as part of its science 
drivers, has strong momentum toward becoming a NASA mission.  It is anticipated that a call for science team
proposals will be announced within the next two years.  I am positioning myself toward being part of
the successful team: I am .  As a member of the JDEM Interim Science Working Group, I designed the
supernova observing strategy planned for WFIRST, and in SNAP I calculated the $S/N$ for the integral
field unit spectroscopy instrument baselined for WFIRST,  

\subsection{Novel Probes of Cosmology}
The future of our field is driven by a combination of community consensus and new unanticipated
ideas that emerge.  The research I have presented to now lay firmly within the
mainstream plan, at least as viewed by the funding agencies, which calls for wide-field imaging surveys
\cite{2013arXiv1309.5386D,2013arXiv1309.5382K}.
Nevertheless, scientific curiosity demands looking into speculative but potentially game-changing
approaches.  I am intrigued with the potential use of time delays of strongly lensed variable in order
to measure cosmic distances.  The a priori unknown intrinsic light curve of the source and differing microlensing
in the path of the two images makes the determination of the time delay a challenge.  I developed a
non-linear regression method to simultaneously model the intrinsic light curve, independent foreground
stochastic processes, and fit the time delay \cite{2013PhRvD..87l3512H}.  With the discovery of more 
systems with DES and LSST, high-resolution modeling of the lens system mass distribution, and accurate
time delay measurements, strongly-lensed
quasars can emerge as the next cross-cutting probe of Dark Energy.

Redshift drift (commonly referred to as the Sandage-Loeb test) provides the most direct  measurement
of the cosmic expansion: measure the change in redshift of an object in some time interval.  The instrumental
and astrophysical challenge
in making such a measurement within a human lifetime
is the need for $10^{-10}$ redshift precisions.  Intriguingly, the velocity resolution needed to measure
cosmological redshift drift is similar to that needed for planet finding through stellar doppler shifts:
interferometric spectrographs designed for planet searches may provide the redshift accuracies for some galaxies
\cite{drift}.
Whether this approach for 
Ongoing spectroscopic surveys such as eBOSS, DESI, and CALIFA can identify 

\subsection{Statistical Methods in Astrophysics}
The scientific program I have described often calls upon regression, classification, dimensionality reduction,
and machine learning: topics within the realm of statistics.  I have applied Gaussian Process
on a diverse range of applications, principal component
analysis (PCA) and expectation maximization PCA \cite{2012PhRvD..85l3530S,2013ApJ...766...84K,
2013PhRvD..87l3512H}, in some cases programming for high performance computers.
My current interest is in non-linear dimensionality reduction: SN~Ia light curves and spectra
are governed by physical processes that are not efficiently described through linear combination.
Supernova.  DESI target selection.  Classification.

eScience Institute.

\subsection{Collaborations}
In the above description of my research interests, I refer to a number of major collaborations of which I am a member.  To
present a clear timeline of my  activities at UW, I review the subset of collaborations with their dates of operation.
\begin{description}
\item[SNfactory (present -- 2015)]  Provides a unique dataset of
spectrophotometric time series of $0.03<z<0.08$ SN~Ia light curves.  The wealth of information
within the data are primed for immediate analysis. The discoveries and results from SNfactory directly tied to the supernova
analysis of DES and LSST.
\item[DES (present -- 2018)] Currently collecting what will be the state-of-the-art deep/wide-field/multi-band
imaging survey.   The ``Stage III'' dark energy experiment, DES will provide the leading supernova
cosmology result in the near future.
\item[OzDES (present -- 2018)] Currently collecting thousands of spectra of interesting targets identified by DES,
filling an essential role in the spectroscopy needs.
\item[DESI (2018-- 2025)]
\item[LSST-DESC (2022 -- 2032)]  The ``Stage IV'' dark energy experiment.
\end{description}

Not listed above are the new collaborations that are critical for the dark energy program.
Collaboration building. WFIRST LSST followup.

\section{Teaching}
In my current position at a National Laboratory I have not had teaching responsibilities.  However,
I have had the opportunity to give guest lectures at UC Berkeley, teach at summer schools, tutor friends
who are continuing their education,, and direct undergraduate and graduate student research:
I look forward to the rewards and challenges of teaching as a professor. 
My teaching philosophy consists of the following tenets:
Mould my approach based on the objectives and motivation of the students in each particular class.
Spend time working problems rather than on theory.
Attack misconceptions head-on.
Actively monitor student response.
Learn what works from great teachers.
Great field for student research. 

On a broader scope, I am interested in increasing underrepresented minorities science majors.
Serving on the Lawrence Berkeley National Laboratory Diversity \& Inclusion Committee, one (of many) things I have
learned is the value of diversity in science.  Science is a social endeavor, increasingly so with
the increasingly larger collaborations needed to address big science.  The informational diversity
coming  from the inclusion of different professional backgrounds and strengths fosters innovation
MLK.
 and the key role all levels of education play.  At each step
in education need to encourage underrepresented minorities.
Office of Minority Affairs \& Diversity Recruitment and Outreach
Hidden bias.



\section{References: Select Publications}
The following references represent a selection from my publication record associated with the research
activities that I will engage in at the University of Washington.

\bibliographystyle{hplain}
\bibliography{pub}
\end{document}
