\documentclass[12pt]{article}

\usepackage{aas_macros}
\usepackage[letterpaper, margin=1in]{geometry}
\usepackage{etoolbox}

\newtoggle{UW}
%\toggletrue{UW}
\iftoggle{UW}{
    \newcommand{\where}{the University of Washington}
}{
\newcommand{\where}{the University of Michigan}
}

\makeatletter
\renewenvironment{thebibliography}[1]
     {%\section*{\refname}% <--- outcommented
      \@mkboth{\MakeUppercase\refname}{\MakeUppercase\refname}%
      \list{\@biblabel{\@arabic\c@enumiv}}%
           {\settowidth\labelwidth{\@biblabel{#1}}%
            \leftmargin\labelwidth
            \advance\leftmargin\labelsep
            \@openbib@code
            \usecounter{enumiv}%
            \let\p@enumiv\@empty
            \renewcommand\theenumiv{\@arabic\c@enumiv}}%
      \sloppy
      \clubpenalty4000
      \@clubpenalty \clubpenalty
      \widowpenalty4000%
      \sfcode`\.\@m}
     {\def\@noitemerr
       {\@latex@warning{Empty `thebibliography' environment}}%
      \endlist}
\makeatother

\date{}

\begin{document}
\title{Research Interests}
\author{Alex Kim}
%\date{\color{green}December 2005}
\maketitle

I am an observational astrophysicist, with interests in
cosmology, supernovae and other transients, and machine learning.
As a new faculty member, I will come to \where{}
primed as a key player in major collaborations, with a
track record in a diverse range of research.  I  value excellence
in teaching and research as part of the student educational experience
and as a means to  further the enrichment of our scientific community.

\section{Research Activity}
My principal field of research is  Type~Ia Supernova (SN~Ia) Cosmology. 
SNe~Ia are exploding stars which have known luminosity at peak brightness, and are  bright enough to be
seen to distances beyond 10 billion light years.  It is through the work
performed as part of my PhD thesis that the accelerating
expansion of the Universe was discovered \cite{1997ApJ...483..565P}.  Today, SNe~Ia remain a key tool used
to measure the cosmic expansion history, which is used both to establish
the  backdrop within which the contents of the Universe evolve,
and to determine the nature of the new physics responsible
for the cosmic acceleration.
SN~Ia surveys are a pillar  in the community plans for the field of cosmology,
as laid out in
the Decadal Survey from the astrophysics and astronomy communities, and the Snowmass
\cite{2013arXiv1309.5386D,2013arXiv1309.5382K} and P5 reports
from the high-energy physics community.  I here detail my research activities,
which are critical to the success of high-redshift supernova cosmology,
and some of my interests beyond the supernova field: in the products of astronomical imaging surveys,
novel probes of cosmology, and advanced statistical tools such as machine learning.

\subsection{Type Ia Supernovae as Cosmological Distance Indicators}
In actuality, the peak luminosities of SNe~Ia are not known, at least to the accuracies necessary
to achieve the lofty objectives of next generation experiments such as the Dark Energy Survey (DES) and the Large Synoptic
Survey Telescope (LSST).  The unique data set of
the Nearby Supernova Factory (SNfactory) collaboration consists of the spectrophotometric
time series, covering pre-maximum to 40-days post-maximum phases,
of  $\sim 150$ nearby ($0.03<z<0.08$) SNe~Ia, obtained with an integral field spectrograph (IFS).
The IFS provides a 3-d view of the sky: a spectrum is obtained for each pixel of a spatial grid covering
the field of view.   This spatial sampling enables the extraction of supernova flux from contaminating host-galaxy
light using  techniques used in image subtraction.
I have used these data  to identify new
empirical
parameters based on synthetic broad-band photometry that reduce distance uncertainties from 7\% to 4.5\% \cite{2013ApJ...766...84K}
while eliminating systematic biases correlated with host-galaxy
properties \cite{2014ApJ...784...51K}.  This work represents only the beginning of the
potential information that can be mined from  SNfactory data: the next order of business
is to include spectroscopic information (rather than limiting ourselves to broadband photometry)
and apply non-linear dimensionality reduction to efficiently find and parameterize
subpopulations   within the SN~Ia family.  We must furnish tools usable by the community so that
knowledge gained from the
exquisite data of the SNfactory can be applied to the lower-resolution, lower signal-to-noise data obtained
by DES and LSST.  To these ends, I am currently mentoring a UC Berkeley graduate student on producing the next-generation empirical
model of SN~Ia luminosities.

\subsection{Measurement of the Expansion History of the Universe With Imaging Surveys  DES and LSST}
The Dark Energy Survey and the Large Synoptic Survey Telescope represent the present and future of
SN~Ia cosmology: they
will generate the dense ground-based Hubble diagram with expanded redshift coverage that will serve as the standard 
through the 2020s.  However,
the most important and challenging product these surveys must provide is not the Hubble curve itself,
but rather its uncertainties: trusted error bars must accompany any experimental evidence for the exclusion
of the Cosmological
Constant as the cause of the acceleration. My research focuses on
minimizing and quantifying the
systematic uncertainties that will dominate the error budgets from DES and LSST \cite{2004MNRAS.347..909K}.  

DES is currently in its second year of a five-year survey running at the 4-m Blanco Telescope in Chile.  It is performing
cadenced observations of 30 square degrees in order to discover and generate SN~Ia light curves out
to $z\sim1$.
The principal challenge for DES (and LSST) is that these surveys do
not provide all the information necessary for precision cosmology.
Unlike previous surveys, the vast majority of SNe~Ia from DES
will not have spectroscopic classification, nor will their
measurements have the signal-to-noise, restframe wavelength coverage, and wavelength resolution that allow  for
the characterization of their SN~Ia subclass.  This classification is necessary to track redshift-dependent evolution of 
the SN~Ia population, an important source of systematic error in the Hubble diagram.  In DES, I  formulated the strategy for performing the cosmology analysis
with a rigorous quantification of systematic uncertainties.  These uncertainties are  derived from a subset of 
DES objects that receive intensive observations beyond what is standard, with
biases measured as the difference between distances inferred with
and without these extra data. 
The DES SN Working Group is beginning its cosmological analyses in earnest at the end
of the 2014B telescope semester and I am now leading efforts to implement my analysis strategy.

LSST  represents the future of ground-based wide-field imaging surveys.
A telescope with an effective 6.7m diameter will operate from 2022-2032, observing the available sky (18,000 square degrees over the course
of a year)
every few days in one of six bands.  In addition, a select $\sim 100$ square degrees will be observed with a higher per-band
cadence of every 4 days
in each of the six bands.
A primary science product of LSST is the discovery and measurement of SN~Ia light curves.
The LSST-Dark Energy Science Collaboration (LSST-DESC) is preparing
to exploit  the public data products that will be
delivered by the  LSST Project.  As co-chair of the LSST-DESC  SN~Ia science working group,
 I have identified and organized this preparatory work \cite{2012arXiv1211.0310L}.
The first order of business is to ensure that LSST itself delivers data that are  close to optimal for our science, though
the selections of filter vendor, survey fields, exposure times, observing cadence, and sky tiling strategy.
Working with the LSST Project\iftoggle{UW}{ (including UW colleagues)}{},
I have developed and implemented survey figures of merit for the resultant supernova science
\cite{LSSTCadence}.  
We are incorporating realistic SN~Ia models  into LSST  data simulators to create the high-fidelity representations
of expected SN signal necessary for survey optimization and algorithm development.
\iftoggle{UW}{Progress on this ongoing work will be
facilitated by my being co-located with key players in the LSST Project.}{}

I look forward to taking advantage of my participation in both DES and LSST through two new projects.
First, despite the temporal ordering of the two experiments,  LSST Data Management 
already has a software base that is competitive with that of DES.  I plan to examine whether the LSST code that extracts photometry simultaneously from mulitband images
of the same source, written for weak gravitational lensing analysis, can be modified for the photometric extraction of supernova signal from DES data.
Second,  a pure and efficient stream of variable objects is an LSST  deliverable that is critical for supernova science.
I plan to use
DES data, as the closest available representation of the Universe that LSST will observe, to test the performance
of the transient pipeline.

\subsection{Spectroscopy That Complements Imaging Surveys}
Imaging surveys such as DES and LSST identify interesting targets through broadband photometry.
Detailed examination of these objects requires spectroscopy. Spectroscopic follow-up is not
built into the DES and LSST Projects themselves, and so must be obtained through independent means.
In DES we compete for public time at
telescopes, and form external collaborations to access other telescopes.  I am a founding member of the OzDES collaboration, specifically formed to obtain spectroscopy
of interesting DES targets using the AAOmega-2dF mounted on the Australian Astronomical Telescope, which
covers 3 square degrees with a 400-fiber multi-object spectrograph.
The sources we observe cater to a broad range of astrophysical interests,
including the classification of live transients, characterization of their host galaxies, confirmation
of strong gravitational lens systems, obtaining redshifts of galaxy clusters, monitoring of active galactic nuclei, photometric-redshift
calibration,
and identification of stars that can be used as
flux standards.  OzDES  made the first identification of a DES-discovered supernova \cite{2013ATel.5568....1C} and continues to churn out transient classifications and thousands of galaxy 
redshifts. My primary interest in OzDES is to obtain a flux-limited sample with which to understand the
underlying population of SNe~Ia  \cite{2006MNRAS.370..933J} and interloping contaminants that constitute the DES supernova
sample.  
Additionally, I am interested in the discovery of new strong gravitational lens systems, and the potential use
of reverberation mapping of active galactic nuclei as a new cosmic distance indicator.


LSST will generate both quiescent and transient targets for spectroscopy.  I am convinced that
a major independent collaboration is called for, one that will have competitive advantage over those with only
the publicly released data.
I am therefore exploring possible resources that can contribute to LSST spectroscopy followup.
Just as OzDES is an almost perfect spectroscopic complement to  DES, 
the Dark Energy Spectroscopic Instrument (DESI)  has a field of view and multiplexing matched to LSST;
DESI is a 5000-fiber multi-object spectrograph covering a $>3\deg$-diameter
field of view that will be mounted on the 4-m Mayall telescope at Kitt Peak National Observatory.   Within the DESI collaboration I have proposed observing LSST targets, both using unused fibers during the
DESI survey and in a dedicated bright-time survey
with unfettered use of all fibers.

SN~Ia spectroscopic subclassification requires an IFS for
the robust photometry of supernovae that cannot be obtained with fiber or slit spectrographs:
indeed, the primary  instrument for supernova {\it photometry} in NASA's WFIRST-AFTA satellite is an IFS.  To take
advantage of LSST discoveries, the community needs a ground-based
network of 2- to 10-m telescopes equipped with  IFS's at the ready for intensive followup observing.  I am currently exploring
programs for the design and building of spectrographs for installation at ground-based
observatories that  can serve as the backbone for a
new collaboration to study SNe~Ia discovered by LSST. 

\subsection{Cosmology with Space-Based Surveys}
The next leap in supernova cosmology requires the stable platform, infrared coverage, and exquisite image
quality afforded by a space telescope.  I have played a leading role in the development
of science and system requirements, and the experimental and survey design of the proposed
and approved space missions SNAP, JDEM, and Euclid
\cite{2006PASP..118..205D, 2011PASP..123..470S, 2013Fourspring, 2014arXiv1409.8562A}.
A  high-redshift supernova program is one of the science 
drivers of
NASA's WFIRST-AFTA satellite.  A call for science team
proposals is anticipated  within the next two years.  As a member of the JDEM Interim Science Working Group, I designed the
supernova observing strategy now adopted by WFIRST, and I calculated the performance for SNAP's IFS which is similar to the one baselined for WFIRST.  Saul Perlmutter and I intend
to remount a prototype IFS built for SNAP as a testbed for
the WFIRST instrument.  Independently, I am contributing a SN~Ia  section to a White Paper 
edited by David Spergel and Bhuvnesh
Jain, which details an expanded science case for WFIRST with specific emphasis
on synergies with LSST.  I am thus positioning myself to be a key member of the winning WFIRST science
team.


\subsection{Novel Cosmological Probes}
Direction in our field is driven by a combination of community consensus and emergent ideas.
The research presented so far  lie  within the
mainstream plan laid out by the funding agencies
\cite{2013arXiv1309.5386D,2013arXiv1309.5382K}.
Nevertheless, scientific curiosity demands looking into speculative but potentially game-changing
techniques, just as we did with our early high-redshift supernova search.
I am intrigued with the potential of measuring cosmic distances using delays in the arrival times of light from multiple images
of strongly lensed quasars.  The a priori unknown intrinsic light curve of the source and differing microlensing
in the image paths complicate the determination of the time delay.  I developed a
non-linear regression method to simultaneously model the intrinsic light curve, independent foreground
stochastic processes, and fit for the time delay \cite{2013PhRvD..87l3512H}.  With the discovery of more 
strongly-lensed quasars with DES and LSST, modeling of their lens-system mass distributions, and accurate
time-delay measurements, strongly-lensed
quasars can emerge as the next cross-cutting probe of Dark Energy, highly complementary with existing methods.

Redshift drift provides the most direct  probe
of the cosmic expansion as the measure of $\dot{z}$, the evolution in redshift of an object over a time interval.
The instrumental
and astrophysical challenge
in making such a measurement within a human lifetime
is that the small signal calls for $10^{-10}$ redshift precisions.  Intriguingly, the wavelength resolution needed to measure
cosmological redshift drift is similar to that needed for the discovery of planets through stellar doppler shifts.
Interferometric spectrographs motivated by planet searches may provide the
requisite redshift accuracies for some emission-line galaxies
\cite{2014arXiv1402.6614K}.
I  am examining whether the astrophysical and instrumental challenges in the feasibility of such a program can be overcome.
Ongoing  emission-line-galaxy  spectroscopic surveys such
as eBOSS, DESI, and the spatially-resolved CALIFA will identify and characterize
galaxies that are well-suited for the measurement, while the building of a prototype spectrograph will demonstrate that
instrumental  accuracies can be achieved.

\subsection{Statistical Methods in Astrophysics}
The scientific program I have described often involves regression, classification, dimensionality reduction,
and machine learning, topics at the intersection of science, statistics, and computing.  In my work,
I have applied Gaussian Process modeling, principal component
analysis (PCA) and expectation maximization PCA \cite{2012PhRvD..85l3530S,2013ApJ...766...84K,
2013PhRvD..87l3512H}, in some cases programming and running on high performance computers.
I am now pursuing non-linear dimensionality reduction, which can more efficiently encapsulate
the physical processes that govern SN~Ia light curves and spectra.
Diffusion maps provide a non-linear coordinate transformation based on the connectivity of a set of points.
I am now starting to explore its use, directing a graduate student on an effort
to  capture diversity in SN~Ia spectral time series,  for the classification of large red galaxies in DES
and for the selection of emission line galaxies that will be targeted by DESI.

\iftoggle{UW}{
My interest in the application of statistics on cosmological data
was born out of interaction with statisticians, computer scientists,
and scientists. I have found that real connections are formed not only through discussion,
but by identifying new projects and the joint writing of new proposals.
I look forward to engaging in
the eScience Institute, which provides an excellent environment to develop new ideas and applications
for machine learning and astrophysics on high-performance computers.}{}


\section{Collaborations}
I have referred to a number of collaborations of which I am a member, whose benefits of participation
and data  I will bring to \where{}.  For clarity, I list these collaborations and their dates of operation.

\begin{description}
\item[SNfactory (present -- 2015)]  Provides a unique dataset of
spectrophotometric time series of $0.03<z<0.08$ SN~Ia light curves.  The wealth of information
within the data are ripe for immediate analysis. The discoveries and results from SNfactory directly tie to the supernova
analysis of DES and LSST.
\item[DES (present -- 2018)] Currently collecting what will be the state-of-the-art deep/wide-field/multi-band
imaging survey.   The ``Stage III'' dark energy experiment, DES will provide the leading supernova
cosmology result in the near future.
\item[OzDES (present -- 2018)] Currently collecting thousands of spectra of  targets identified by DES,
filling an essential role in our spectroscopy needs.
\item[DESI (2018-- 2025)]  Primarily a galaxy and quasar redshift survey, the instrumentation is well-matched for
 spectroscopy of targets identified by LSST. Passed the first DOE funding review.
\item[LSST-DESC (2022 -- 2032)]  The ``Stage IV'' dark energy experiment that will produce large streams of
transient discoveries and light curves, including SNe~Ia over a broad redshift range.
\end{description}

I am working to build new collaborations that are critical for the dark energy program, 
to
spectroscopically follow-up LSST targets, and to be a member of the WFIRST science team.


\section{References: Select Publications}
The following references represent a selection from my publication record associated with the research
activities that I will engage in at \where{}.

\bibliographystyle{hplain}
\bibliography{pub}
\end{document}
